\documentclass[./thesis.tex]{subfiles}

\begin{document}

% {\fontsize{13}{12} \selectfont
% Kỹ thuật phân vùng ngữ nghĩa có nhiều ứng dụng trong lĩnh vực đồ họa kỹ thuật số, ...}

{\fontsize{13}{12} \selectfont
Để giải quyết vấn đề rác thải trên đường phố hiện nay, việc phát hiện và phân loại rác thải là tiền đề để giúp mỗi địa phương đưa ra giải pháp xử lí, tái chế rác thải một các hợp lí. Do đó, nghiên cứu này sẽ giải quyết vấn đề trên bằng việc xây dựng hệ thống nhận dạng và phân loại rác thải. Nghiên cứu sẽ sử dụng một số mô hình học sâu (...), sau đó tìm ra mô hình tối ưu nhất để làm mô hình xây dụng hệ thống. Bộ dữ liệu Taco gồm hơn 2000 hình ảnh kết hợp với tập dữ liệu hơn 1000 hình ảnh rác được thu thập ở một số tỉnh Đồng bằng sông Cửu Long sẽ được sử dụng trong nghiên cứu.


Cuối cùng, xây dựng ứng dụng nhận hình ảnh đầu vào là hình ảnh và đầu ra là hình ảnh đã được nhận dạng và phân loại rác.
}
\bigskip

% {\bf Từ khóa:} \textit{ X-quang, hình ảnh y tế, mạng tích chập đầy đủ, phát hiện gãy xương, YOLOv4, U-Net.}

\end{document}
\documentclass[./thesis.tex]{subfiles}

\begin{document}

% {\fontsize{13}{12} \selectfont
% Kỹ thuật phân vùng ngữ nghĩa có nhiều ứng dụng trong lĩnh vực đồ họa kỹ thuật số, ...}

{\fontsize{13}{12} \selectfont

Hiện nay các tỉnh thành ở Việt Nam đang triển khai dự án thành phố thông minh, ngoài việc giám sát giao thông, cải thiện dịch vụ công, phát triển y tế, hệ thống rác thải thông minh cũng đóng vai trò quan trọng để cải thiện đời sống đô thị.
Việc phát hiện và phân loại rác thải trên đường phố giúp địa phương nắm bắt tình hình ô nhiễm rác ở khu vực, từ đó có biện pháp xử lí rác thải kịp thời, tránh tình trạng rác tồn đọng và hình thành các bãi rác bất hợp pháp gây ảnh hưởng đến sức khỏe người dân.
Nghiên cứu này sẽ giải quyết vấn đề trên bằng việc xây dựng hệ thống nhận dạng và phân loại rác sử dụng một số mô hình phát hiện mới hiện nay như YOLOv7, YOLOv8, SSD MobileNetv2.
Bộ dữ liệu khoảng 4.000 hình ảnh được chia thành bốn lớp: kim loại, giấy, nhựa - nilon và rác khác là sự kết hợp của các tập dữ liệu TACO, TrashNet và dữ liệu hơn 800 hình ảnh rác được thu thập ở một số tỉnh Đồng bằng sông Cửu Long.
Kết quả thực nghiệm cho thấy YOLOv8 đạt kết quả tốt hơn hai mô hình còn lại với mAP50 cho phát hiện rác trên đường phố là $73\%$ (một lớp) và $44,6\%$ (bốn lớp).
Cuối cùng, nghiên cứu có so sánh kết quả với một số nghiên cứu khác và rút ra những vấn đề quan trọng để xây dựng hệ thống.
Nghiên cứu cũng cho thấy việc bổ sung dữ liệu các hình nền không chứa đối tượng là rác đã giúp cải thiện khả năng nhận dạng của mô hình trong việc phát hiện rác thải trong tự nhiên.

}
\bigskip

{\bf Từ khóa:} \textit{Rác, phân loại rác, YOLO, SSD, MobileNet, TACO, TrashNet}

\end{document}
\documentclass[./thesis.tex]{subfiles}

\begin{document}

% {\fontsize{13}{12} \selectfont
% Kỹ thuật phân vùng ngữ nghĩa có nhiều ứng dụng trong lĩnh vực đồ họa kỹ thuật số, ...}

{\fontsize{13}{12} \selectfont
Để giải quyết vấn đề rác thải trên đường phố hiện nay, việc phát hiện và phân loại rác thải là tiền đề để giúp mỗi địa phương đưa ra giải pháp xử lí, tái chế rác thải phù hợp.
Do đó, nghiên cứu này sẽ giải quyết vấn đề trên bằng việc xây dựng hệ thống nhận dạng và phân loại rác thải. Nghiên cứu sẽ sử dụng một số mô hình phát hiện một giai đoạn
như YOLOv7 \cite{wang2022yolov7}, YOLOv8 \cite{YOLOv8}, SSD MobileNetv2 \cite{Liu_2016} \cite{sandler2019mobilenetv2} sau đó tìm ra mô hình tối ưu nhất để làm mô hình xây dụng hệ thống.
Bộ dữ liệu khoảng 4000 hình ảnh là sự kết hợp của các tập dữ liệu Taco \cite{proença2020taco}, TrashNet \cite{yang2016classification} và tập dự liệu tự thu thập hơn 800 hình ảnh rác được thu thập ở một số tỉnh Đồng bằng sông Cửu Long sẽ được sử dụng trong nghiên cứu.
Nghiên cứu cũng cho thấy việc bổ sung dữ liệu các hình nền không chứa vật thể giúp cải thiện khả năng nhận dạng của mô hình trong việc phát hiện rác thải trong tự nhiên.
}
\bigskip



{\bf Từ khóa:} \textit{Rác, phân loại rác, YOLO, YOLOv7, YOLOv8, SSD MobileNetv2, Taco, TrashNet}

\end{document}
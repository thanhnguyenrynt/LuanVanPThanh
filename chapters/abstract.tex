\documentclass[./thesis.tex]{subfiles}

\begin{document}
% {\fontsize{13}{12} \selectfont
% The semantic image segmentation technique has many applications in digital graphics, ...}

{\fontsize{13}{12} \selectfont
To address the issue of street littering, waste detection and classification are crucial steps for each locality to develop appropriate waste treatment and recycling solutions.
Therefore, this study aims to tackle this problem by constructing a waste recognition and classification system. The study will employ various one-stage detection models such as Yolov7 \cite{wang2022yolov7}, Yolov8 \cite{Yolov8}, SSD MobileNetv2 \cite{Liu_2016} \cite{sandler2019mobilenetv2} to identify the most optimal model for system development.
A dataset of approximately 4000 images will be utilized, combining the Taco \cite{proença2020taco}, TrashNet \cite{yang2016classification} datasets, and a self-collected dataset of over 800 waste images gathered from provinces in the Mekong Delta region.
The study also demonstrates that augmenting the dataset with background images without objects improves the model's recognition ability in detecting waste in natural environments.
}
\bigskip

{\bf Keywords:} \textit{Trash, trash detection, waste, Yolo, Yolov7, Yolov8, SSD MobileNetv2, Taco, TrashNet.}

\end{document}
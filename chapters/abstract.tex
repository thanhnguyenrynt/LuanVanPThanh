\documentclass[./thesis.tex]{subfiles}

\begin{document}
% {\fontsize{13}{12} \selectfont
% The semantic image segmentation technique has many applications in digital graphics, ...}

{\fontsize{13}{12} \selectfont
% Wrist fractures are commonly diagnosed using X-ray imaging, supplemented by magnetic resonance imaging and computed tomography when required. Radiologists can sometimes overlook fractures because they are difficult to spot. In contrast, some fractures can be easily spotted and only slow down the radiologists because of the reporting systems. 
% We propose a machine learning model based on the YOLOv4 method that can help solve these issues. The rigorous testing of three levels showed that the YOLOv4-based model obtained significantly better results in comparison to the state-of-the-art method based on the U-Net model. We leverage the public dataset which containing more than 20,000 X-ray images to conduct the experiments. The YOLOv4 achieves an accuracy of 0.89871, recall of 0.89871, precision of 0.90369, and $F_1$ of 0.89997, outperforms the U-Net model.

% Finally, we compared our work with other related work and discussed what to consider when building an ML-based predictive model for wrist fracture detection.  
}
\bigskip

% {\bf Keywords:} \textit{ X-ray, medical image, fully convolutional network, fracture detection, U-Net, Yolo.}

\end{document}
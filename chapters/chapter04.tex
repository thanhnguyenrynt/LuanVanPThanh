\documentclass[../the.tex]{subfiles}


\begin{document}
{\fontsize{13}{12} \selectfont
Trong chương này tiền hành đánh giá và thảo luận kết quả. Các mô hình sẽ được đánh giá dựa vào số lượng tham số sử dụng, tốc độ FPS và độ chính xác mAP50.
Nghiên cứu sẽ tiến hành đánh giá các mô hình dựa vào bộ dữ liệu \#1, \#2, \#3 đã đề cập ở phần \ref{sec:dataset} lần lượt được gọi là Thực nghiệm 1, 2, 3.
Các tập xác thực được tạo từ $20\%$ từ bộ dữ liệu, tập thử nghiệm ($5\%$) để đánh giá kết quả trên hình ảnh thực tế ở đường phố được lấy từ tập dữ liệu Taco và tự thu thập.
Các mô hình được huấn luyện trên máy tính cấu hình GPU GTX 3050 với bộ nhớ 8GB.
}

\section{Thực nghiệm 1 - Đánh giá mô hình dựa vào phát hiện rác}
\begin{table}[ht!]
    \centering
    \caption{Các tham số tăng cường dữ liệu cho các mô hình}
    \begin{tabular}{|l|p{7cm}|c|}
        \hline
        \textbf{Tên}
                   & \textbf{Mô tả}
                   & \textbf{(\%)}
        \\ \hline
        Lật ngang  & Lật hình ảnh từ trái sang phải, tăng tính đa dạng dữ liệu                                                                                    & 50 \\  \hline
        Lật dọc    & Lật ngược hình ảnh, không ảnh hưởng đến đặc điểm của đối tượng                                                                               & 50 \\  \hline
        Xoay       & Xoay hình ảnh ngẫu nhiên, cải thiện khả năng nhận dạng vật thể ở nhiều hướng                                                                 & 50 \\  \hline
        Xê dịch    & Dịch hình ảnh theo chiều ngang và chiều dọc bằng một phần kích thước hình ảnh, hỗ trợ học cách phát hiện các vật thể nhìn thấy được một phần & 10 \\  \hline
        Scale      & Phóng to, thu nhỏ hình ảnh, mô phỏng hình ảnh với tỉ lệ khác nhau so với hình ảnh chụp                                                       & 50 \\  \hline
        Độ bão hòa & Thay đổi độ bão hòa của hình ảnh một phần, ảnh hưởng đến cường độ màu sắc. Hữu ích cho việc mô phỏng các điều kiện môi trường khác nhau      & 70 \\ \hline
        Độ sáng    & Sửa đổi độ sáng của hình ảnh theo một phần nhỏ, giúp mô hình hoạt động tốt trong nhiều điều kiện ánh sáng khác nhau                          & 40 \\ \hline
    \end{tabular}
    \label{tab:thamso}
\end{table}
{\fontsize{13}{12} \selectfont
Thực nghiệm 1 với mục tiêu đánh giá các mô hình dựa trên việc phát hiện vật thể có phải là rác hay không, trong tập dữ liệu có một nhãn duy nhất là rác.
Các mô hình được so sánh trong nhiệm vụ phát hiện đối tượng, các phương pháp tăng cường dữ liệu được cài đặt cho các mô hình là như nhau được thể hiện ở bảng \ref{tab:thamso}.
Các tham số còn lại cài đặt theo mặc định của mô hình, như tăng cường mosiac ở Yolov8, focal loss ở Yolov7 mục đích đánh giá đặc trưng riêng của các mô hình.

}
\begin{table}[h!]
    \centering
    \caption{Kết quả thực hiện Thực nghiệm 1 với tập xác thực}
    \begin{tabular}{|l|c|c|c|}
        \hline
        \textbf{Mô hình}  & \textbf{Precision} & \textbf{Recall} & \textbf{mAP50} \\ \hline
        SSD   MobileNetv2 & 0.841               & 0.595           & 0.657          \\ \hline
        Yolov7-tiny       & 0.81               & 0.699           & 0.788          \\ \hline
        Yolov8n           & \textbf{0.873}     & \textbf{0.727}  & \textbf{0.842} \\ \hline
    \end{tabular}
    \label{tab:thucnghiem1.1}
\end{table}

\begin{table}[h!]
    \centering
    \caption{Kết quả thực hiện Thực nghiệm 1 với tập thử nghiệm}
    \begin{tabular}{|l|c|c|c|c|}
        \hline
        \textbf{Mô hình}  & \textbf{Precision} & \textbf{Recall} & \textbf{mAP50} & \multicolumn{1}{l|}{\textbf{FPS}} \\ \hline
        SSD MobileNetv2 & 0.712               & 0.364           & 0.427          & 105                               \\ \hline
        Yolov7-tiny       & 0.723              & 0.515           & 0.62          & \textbf{136}                      \\ \hline
        Yolov8n           & \textbf{0.73}     & \textbf{0.606}  & \textbf{0.713} & 125                               \\ \hline
    \end{tabular}
    \label{tab:thucnghiem1.2}
\end{table}

{\fontsize{13}{12} \selectfont
    Các mô hình được huấn luyện bằng phương pháp học chuyển giao dựa trên tập dữ liệu coco, quá trình huấn luyện là 100 kỷ nguyên, ngưỡng tin cậy để đánh giá là 0.3.
    Kết quả đánh giá của Thực nghiệm 1 được thể hiện ở Bảng \ref{tab:thucnghiem1.1} 
    và Bảng \ref{tab:thucnghiem1.2} với tập dữ liệu xác thực và tập dữ liệu thực nghiệm cho thấy mô hình 
    Hình \ref{fig:thucnghiem1} cho thấy Yolov8 đạt kết quả tốt hơn so với hai mô hình còn lại. Đối với tập xác thực, mô hình SSD MobileNetv2 có khả năng nhận dạng đúng đối tượng tốt hơn so với Yolov7-tiny, tuy nhiên Recall của mô hình lại kém so với hai mô hình họ Yolo.
    Đối với tập dữ liệu thử nghiệm, mô hình Yolov8n vẫn vượt rội so với những mô hình còn lại, chỉ số Precision của các mô hình không có sự chênh lệnh lớn (khoảng 2\%), tuy nhiên chỉ số Recall của tất cả giảm đáng kể, giảm nhiều nhất là mô hình SSD MobileNetv2 chỉ còn 36.4\%. 
    Kết quả chênh lệch giữa tập dữ liệu xác thực và tập dữ liệu thực nghiệm khoảng 10 - 20\% cho thấy việc phát hiện rác ở tự nhiên khó hơn điều kiện lí tưởng bởi chịu ảnh hưởng của môi trường xung quanh vật thể.
    Các Hình \ref{fig:thucnghiem1.2} và Hình \ref{fig:thucnghiem1.3} thể hiện khả năng nhận dạng của mô hình Yolov8.

}

\begin{figure}[H]
    \centering
    \includegraphics[width=0.8\textwidth]{thucnghiem1.png}
    \caption{Biểu đồ kết quả thực nghiệm 1}
    \label{fig:thucnghiem1}
\end{figure}

\begin{figure}[H]
	\centering
	\includegraphics[width=0.8\textwidth]{matrix_tn1.png}
	\caption{Ma trận nhầm lẫn trên tập dữ liệu xác thực của mô hình Yolov8n}
	\label{fig:thucnghiem1.2}
\end{figure}

\begin{figure}[H]
	\centering
	\includegraphics[width=0.8\textwidth]{th1_data.png}
	\caption{Kết quả dự đoán trên tập xác thực của mô hình Yolov8n. Bên trái là hình được gán nhãn, bên phải là hình được phát hiện bởi mô hình}
	\label{fig:thucnghiem1.3}
\end{figure}

\section{Thực nghiệm 2 và 3 - Đánh giá mô hình dựa vào phát hiện và phân loại rác}
 {\fontsize{13}{12} \selectfont
  Các mô hình vẫn sử dụng tham số để tăng cường dữ liệu giống như ở \ref{tab:thamso}, thực hiện trên tập dữ liệu \#2 và \#3 như đã đề cập ở phần \ref{sec:dataset}.
  Đánh giá các mô hình dựa vào chỉ số mAP50 trên từng loại rác và trung bình của tất cả, kết quả đánh giá được thể hiện ở.
 }

\bigskip

{\fontsize{13}{12} \selectfont
    Mô hình SSD MobileNetv2 tỏ ra không hiệu quả so với hai mô hình còn lại trong Thực nghiệm 1, vì vậy trong Thực nghiệm 2 và 3, nghiên cứu tập trung vào Yolov7-tiny và Yolov8n.
    Các mô hình nhận dạng tốt ở lớp kim loại và thấp nhất ở lớp khác,
    điều này có thể giải thích bằng tính đặc trưng của lớp khác quá nhiều, đa dạng về hình thể, đa phần là các rác nhỏ và khó xác định.
    Trong khi đó lớp kim loại dễ xác định vì chủ yếu là các lon nước với kích thước và hình dạng đặc trưng. (... note Bổ sung thêm đặc tính các lớp khác).
    Hình (matrix note) cho thấy việc các hình nền dễ bị nhận dạng trở thành các vật thể rác, trong đó có thể gồm các vật thể như đá, các biển báo, vũng nước.
}
\begin{table}[h!]
    \centering
    \caption{Kết quả thực hiện Thực nghiệm 2 với tập xác thực}
    \begin{tabular}{|l|c|c|c|l|l|}
        \hline
        \textbf{Mô hình}  & \textbf{Kim loại}          & \textbf{Giấy}              & \textbf{Nhựa - nilon}      & \multicolumn{1}{c|}{\textbf{Khac}} & \multicolumn{1}{c|}{\textbf{Tất cả}} \\ \hline
        SSD   MobileNetv2 & 0.711                      & 0.707                      & 0.601                      & 0.459                              & 0.62                               \\ \hline
        Yolov7-tiny       & 0.721                      & 0.716                      & 0.598                      & 0.498                              & 0.633                                \\ \hline
        Yolov8n           & 0.759                      & 0.703                      & 0.652                      & 0.581                              & 0.674                                \\ \hline
    \end{tabular}
    \label{tab:thucnghiem2.1}
\end{table}

\begin{table}[h!]
    \centering
    \caption{Kết quả thực hiện Thực nghiệm 2 với tập thử nghiệm}
    \begin{tabular}{|l|c|c|c|l|l|}
        \hline
        \textbf{Mô hình}  & \textbf{Kim loại}          & \textbf{Giấy}              & \textbf{Nhựa - nilon}      & \multicolumn{1}{c|}{\textbf{Khac}} & \multicolumn{1}{c|}{\textbf{Tất cả}} \\ \hline
        SSD   MobileNetv2 & 0.711                      & 0.707                      & 0.601                      & 0.459                              & 0.6195                               \\ \hline
        Yolov7-tiny       & 0.721                      & 0.716                      & 0.598                      & 0.498                              & 0.633                                \\ \hline
        Yolov8n           & 0.759                      & 0.703                      & 0.652                      & 0.581                              & 0.674                                \\ \hline
    \end{tabular}
    \label{tab:thucnghiem2.2}
\end{table}

\section{Thảo luận} 
{\fontsize{13}{12} \selectfont 
Các kết quả được so sánh cho ta thấy cùng một kết quả: mô hình Yolov8 hoạt động tốt hơn hai mô hình còn lại.
Từ đánh giá việc phát hiện đối tượng (phát hiện một lớp) đến việc phát hiện và phân loại rác thành bốn lớp theo thành phần.
Các mô hình luôn gặp khó khăn khi phát hiện vật thể với lớp rác khác vì có kích thước nhỏ và tính đa dạng. 
Điều đó cho thấy việc gán nhãn và định nghĩa các rác cụ thể thành các lớp đặc trưng mang ý nghĩa quan trọng,
tương tự như việc chuyển đổi các lớp ở dữ liệu Taco thành các lớp ở dữ liệu nghiên cứu. 
Thực nghiệm 3 bổ sung thêm dữ liệu từ các nền đã tăng độ chính xác của mô hình, giảm dương tính giả, giúp mô hình phát hiện chính xác các vật thể là rác thay vì nền.
}

\bigskip

{\fontsize{13}{12} \selectfont 
Tổng kết lại, cả ba thực nghiệm đã cho thấy mô hình Yolov8 có hiệu suất vượt trội ở độ chính xác 
mà còn có số lượng tham số ít hơn giúp việc huấn luyện nhanh, phù hợp với việc cải thiện độ chính xác của mô hình khi dữ liệu được tăng cường.
Nghiên cứu cho thấy tầm quan trọng của việc kết hợp dữ liệu, gán nhãn phân loại và áp dụng các hình không có vật thể giúp cải thiện hiệu suất của mô hình khi thực nghiệm ngoài tự nhiên.

}

\end{document}
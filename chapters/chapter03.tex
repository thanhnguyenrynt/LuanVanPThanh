\documentclass[../the.tex]{subfiles}
\begin{document}
{\fontsize{13}{12} \selectfont

Trong phạm vi nghiên cứu này sẽ sử dụng mô hình YOLOv7, YOLOv8, SSD MobileNet. Mục \ref{sec:dataset} mô tả về tập dữ liệu. Mục \ref{sec:model} giới thiệu về các mô hình được sử dụng và sơ đồ hệ thống ở Mục \ref{sec:sodo}.

}
\section{Tập dữ liệu}
\label{sec:dataset}

{\fontsize{13}{12} \selectfont

	Trong phạm vi nghiên cứu, các tập dữ liệu sử dụng sẽ là TrashNet \cite{yang2016classification}, TACO \cite{proença2020taco} và bổ sung tập dữ liệu tự thu thập được ở ĐBSCL.
	Tham khảo việc phân loại rác của Yan \cite{yang2016classification} và quá trình thu thập dữ liệu thực tế nhận thấy các loại rác bằng thủy tinh ở môi trường bên ngoài rất thường là các mảnh kính trong suốt nhìn thấy nên hoặc có độ phản chiếu ánh sáng cao nên rất khó phát hiện, vì vậy các thủy tinh sẽ được gom vào các loại rác khác.
	Lớp giấy và thùng giấy sẽ được gom lại vì cùng chất liệu. Cuối cùng mô hình sẽ phân các loại ra làm các lớp:
	\begin{itemize}
		\item Kim loại: các nắp bia, các vỏ bình làm bằng kim loại.
		\item Nhựa - nilon rất phổ biến, gồm các vật liệu làm từ nhựa như ly nhựa, các túi nilon.
		\item Giấy: vật liệu từ giấy, các thùng giấy, vỏ gói thuốc lá, hộp giấy.
		\item Rác khác là các loại rác còn lại với từng loại xuất hiện ít hoặc khó phát hiện như thủy tinh, các vỏ gói đa sắc,
		      mút, vải,\dots
	\end{itemize}
}

\subsection{TrashNet}
\label{sec:trashnet}
{\fontsize{13}{12} \selectfont

	TrashNet là tập dữ liệu được giới thiệu trong bài nghiên cứu của \cite{yang2016classification} bao gồm các lớp và số lượng như Bảng \ref{tab:dataset}. Tất cả hình ảnh được chụp bằng điện thoại Iphone7 sử dụng ánh sáng mặt trời hoặc ánh sáng phòng, các đối tượng được chụp trong nền trắng hoặc bao quát toàn bộ khung hình, một số hình ảnh ở tập TrashNet ở hình
	\ref{fig:dataset_0}.
	Do mục đích ban đầu của tập dữ liệu dùng để phân lớp nên nghiên cứu phải thực hiện gán hộp giới hạn cho bộ dữ liệu để phù hợp với nhu cầu phát hiện đối tượng. Tập dữ liệu TrashNet bao gồm các đối tượng có kích thước lớn và rõ ràng, mục đích sử dụng tăng độ nhận dạng cho mô hình. Bảng \ref{tab:dataset1} thể hiện số lượng đối tượng sau khi đã gán hộp giới hạn cho tập dữ liệu, hình
	\ref{fig:dataset_1} mô tả tập dữ liệu TrashNet khi được gán hộp giới hạn.

}

\begin{table}[!ht]
	\centering
	\begin{threeparttable}
		\caption{Số lượng ảnh theo lớp của tập dữ liệu TrashNet}

		\begin{tabular}{llr}
			\hline
			\multicolumn{1}{l}{
				\textbf{\#}}
			 & \multicolumn{1}{l}{\textbf{Lớp}}
			 & \multicolumn{1}{r}{\textbf{Số lượng ảnh}} \\
			\hline

			1
			 & Cardboard
			 & 403                                       \\
			\hline

			2
			 & Paper
			 & 594                                       \\
			\hline

			3
			 & Glass
			 & 501                                       \\
			\hline

			4
			 & Plastic
			 & 482                                       \\
			\hline

			5
			 & Metal
			 & 410                                       \\
			\hline

			6
			 & Trash
			 & 137                                       \\
			\hline


			\textbf{Tổng cộng}
			 &
			 & 2524                                      \\
			\hline
		\end{tabular}
		\label{tab:dataset}
	\end{threeparttable}
\end{table}

\begin{figure}[H]
	\centering
	\includegraphics[width=1\textwidth]{trashnet_sample.png}
	\caption{Ví dụ về hình ảnh của tập dữ liệu TrashNet}
	\label{fig:dataset_0}
\end{figure}

\begin{table}[!ht]
	\centering
	\begin{threeparttable}
		\caption{Số lượng ảnh theo lớp của tập dữ liệu TrashNet khi được gán hộp giới hạn}

		\begin{tabular}{llr}
			\hline
			\multicolumn{1}{l}{
				\textbf{\#}}
			 & \multicolumn{1}{l}{\textbf{Lớp}}
			 & \multicolumn{1}{r}{\textbf{Số lượng vật thể}} \\
			\hline

			1
			 & Cardboard
			 & 404                                           \\
			\hline

			2
			 & Paper
			 & 601                                           \\
			\hline

			3
			 & Glass
			 & 509                                           \\
			\hline

			4
			 & Plastic
			 & 479                                           \\
			\hline

			5
			 & Metal
			 & 410                                           \\
			\hline

			6
			 & Trash
			 & 149                                           \\
			\hline


			\textbf{Tổng cộng}
			 &
			 & 2552                                          \\
			\hline
		\end{tabular}
	\label{tab:dataset1}
	\end{threeparttable}
\end{table}

\begin{figure}[H]
	\centering
	\includegraphics[width=1\textwidth]{trashnet_sample2.png}
	\caption{Ví dụ về hình ảnh của tập dữ liệu TrashNet được gán hộp giới hạn}
	\label{fig:dataset_1}
\end{figure}

\subsection{TACO}
\label{sec:TACO}
{\fontsize{13}{12} \selectfont

	TACO là một tập dữ liệu hình ảnh về chất thải trong tự nhiên. Tập dữ liệu chứa các hình ảnh về rác được chụp trong nhiều môi trường khác nhau, từ những bãi biển đến đường phố London. Những hình ảnh này được gắn nhãn và phân đoạn thủ công để huấn luyện và đánh giá các thuật toán phát hiện đối tượng.
	Hiện tại bộ dữ liệu có 1.500 ảnh với 4.784 vật thể
	và 3.918 ảnh mới cần được gán nhãn. Một số hình ảnh của tập dữ liệu ở Hình \ref{fig:dataset_taco}.

}

\begin{figure}[H]
	\centering
	\includegraphics[width=1\textwidth]{Taco.png}
	\caption{Ví dụ về hình ảnh của bộ dữ liệu TACO \cite{proença2020taco}}
	\label{fig:dataset_taco}
\end{figure}


{\fontsize{13}{12} \selectfont
Bộ dữ liệu TACO được cung cấp theo chuẩn json của COCO (xem Hình \ref{fig:coco_full_format}).
Các trường dữ liệu cung cấp thông tin về giấy phép, đường dẫn các hình ảnh, thông tin về danh mục, ngữ cảnh và đặc biệt là thông tin gán nhãn của đối tương được lưu ở trường \textbf{annotations}.
Hình \ref{fig:taco_format} mô tả các chi tiết về một đối tượng được gán nhãn gồm hình ảnh chứa đối tượng, nhãn, diện tích, đa giác phân đoạn và hộp giới hạn. Hộp giới hạn được lưu theo cấu trúc $[x,y,width,height]$ lần lượt là tọa độ điểm, chiều rộng, chiều cao.

}

\begin{figure}[H]
	\centering
	\includegraphics[width=0.3\textwidth]{COCO_format.png}
	\caption{Cấu trúc tệp theo chuẩn COCO của dữ liệu TACO}
	\label{fig:coco_full_format}
\end{figure}

\begin{figure}[H]
	\centering
	\includegraphics[width=0.5\textwidth]{COCO_full_format.png}
	\caption{Cấu trúc lưu trữ thông tin một đối tượng được gán nhãn trong bộ dữ liệu TACO}
	\label{fig:taco_format}
\end{figure}

{\fontsize{13}{12} \selectfont

Bộ dữ liệu gồm 28 danh mục lớn  (xem Hình \ref{fig:dataset_taco_1_a}) và 60 danh mục nhỏ (xem Hình \ref{fig:dataset_taco_1_b}) với
6 loại nền là rác, thảm cỏ, nước, trong nhà, vỉa hè, cát đá (xem Hình \ref{fig:dataset_taco_1_c}). Bộ dữ liệu TACO thể hiện sự đa dạng trong từng loại rác,
tuy nhiên đó cũng là sự hạn chế khi có những lớp có quá ít dữ liệu như Carded blister pack (1 đối tượng), Battery (2 đối tượng), và các lớp có quá nhiều đối tượng như Cigarette (667 đối tượng) và Unlabeled litter (516 đối tượng).

}

\begin{figure}[H]
	\centering
	\subfloat[\centering {\fontsize{11}{10} \selectfont Danh mục lớn }]{{\includegraphics[width=7cm]{taco_super.png} \label{fig:dataset_taco_1_a}}}%
	\qquad
	\subfloat[\centering {\fontsize{11}{10} \selectfont Danh mục nhỏ}]{{\includegraphics[width=7cm]{taco_cat.png} \label{fig:dataset_taco_1_b}}}%
	\qquad
	\subfloat[\centering {\fontsize{11}{10} \selectfont Tỉ lệ nền}]{{\includegraphics[width=10cm]{taco_bg.png} \label{fig:dataset_taco_1_c}}}%
	\caption{Các thống kê theo danh mục lớn, danh mục nhỏ, tỉ lệ nền của bộ dữ liệu TACO}%
	\label{fig:dataset_taco_1}
\end{figure}



\subsection{Dữ liệu tự thu thập}
\label{sec:own}
{\fontsize{13}{12} \selectfont

	Hình \ref{fig:dataset_own} là bộ dữ liệu được thực hiện lấy mẫu bằng máy ảnh điện thoại ở các tỉnh thành ở ĐBSCL như Cần Thơ, Vĩnh Long và chủ yếu ở Trà Vinh nhầm phản ánh thực tế tình trạng rác ở địa phương.
	Dữ liệu thu thập gồm 663 hình ảnh với 1.766 đối tượng được chia cho bốn lớp kim loại, nhựa - nilon, giấy, rác khác với trung bình 2,6 đối tượng mỗi hình ảnh (xem Bảng \ref{tab:datasetown}).
	Phần lớn là các hộp thức ăn giấy, các bọc nilon, ly nhựa, túi rác, khẩu trang giấy và các phần rác cũ không có hình dạng cố định.
	Việc thu thập thêm dữ liệu nhầm giúp mô hình học tập và cải thiện khả năng phát hiện đúng thực tế. Trong quá trình thực hiện thu thập dữ liệu gặp những vấn đề như sau:

	\begin{itemize}
		\item Các rác chồng chéo khó xác định hộp giới hạn.
		\item Tình trạng rác cũ, rác nhỏ bị ẩn vào nền rất khó phát hiện, chủ yếu là các rác nilon đã bắt đầu phân hủy.
		\item Xuất hiện các đối tượng rác có thể gồm nhiều nhãn. Ví dụ như túi nilon chứa bọc giấy, lon kim loại, các gói sắc nét hộp thức ăn nhựa trong suốt chứa rác hữu cơ.
		\item Phần rác kim loại xuất hiện ít, vì hầu hết được người dân xử lí chủ động trước. Tuy nhiên phần rác kim loại vẫn có đặc tính dễ phát hiện và bổ sung dữ liệu từ TACO \cite{proença2020taco} và TrashNet.
		\item Các loại còn lại được gom vào rác khác sẽ giảm khả năng phân lớp vì độ đa dạng cao.
	\end{itemize}

}

\begin{table}[!ht]
	\centering
	\begin{threeparttable}
		\caption{Số lượng ảnh, đối tượng và tỉ lệ theo lớp của tập dữ liệu tự thu thập}
		\begin{tabular}{lrrwr{3cm}}
			\hline
			\multicolumn{1}{l}{\textbf{Lớp}}
			                    & \multicolumn{1}{r}{\textbf{Số hình}}
			                    & \multicolumn{1}{r}{\textbf{Số đối tượng}}
			                    & \multicolumn{1}{r}{\textbf{Tỉ lệ}}
			\\
			\hline

			Nhựa - Nilon        & 392                                       & 761 & \t43,1\% \\
			\hline

			Giấy                & 339                                       & 552 & 31,3\%   \\
			\hline

			Rác khác            & 209                                       & 350 & 19,8\%   \\
			\hline

			Kim loại            & 60                                        & 103 & 5,8\%    \\
			\hline
			Ảnh không đối tượng & 200                                       & 0   & 0        \\
			\hline
		\end{tabular}
	\label{tab:datasetown}
	\end{threeparttable}
\end{table}

{\fontsize{13}{12} \selectfont

Bảng \ref{tab:datasetown} thể hiện rác thải nhựa-nilon chiếm phần lớn các loại rác ở đường phố hiện nay và rác kim loại rất ít chỉ khoảng 5,8\% dữ liệu thu thập được.
Việc mất cân bằng dữ liệu ở tập dữ liệu tự thu thập sẽ được bổ sung bằng các tập dữ liệu của TACO và TrashNet, từ đó tạo ra bộ dữ liệu để huấn luyện và kiểm thử.
Ngoài ra nghiên cứu còn thu thập thêm 200 hình ảnh nền để tăng cường dữ liệu, đặt ra vấn đề các ảnh nền không có chú thích (background) có tăng độ chính xác của mô hình bằng cách giảm việc nhận dạng sai các đối tượng không phải là rác hay không.

}


\begin{figure}[H]
	\centering
	\includegraphics[width=0.75\textwidth]{data_own.png}
	\caption{Các hình ảnh dữ liệu thu thập khi được chú thích hộp giới hạn}
	\label{fig:dataset_own}
\end{figure}

\begin{figure}[H]
	\centering
	\includegraphics[width=0.75\textwidth]{hinh_background.png}
	\caption{Các hình nền không có chú thích để tăng cường dữ liệu}
	\label{fig:dataset_bg}
\end{figure}

\subsection{Dữ liệu sử dụng cho mô hình}
{\fontsize{13}{12} \selectfont

	Trong phạm vi nghiên cứu sẽ sử dụng kết hợp giữa các bộ dữ liệu được nêu ở Mục \ref{sec:trashnet}, \ref{sec:TACO}, \ref{sec:own} để tạo ra các bộ dữ liệu phù hợp với các mục đích huấn luyện.
	Bảng \ref{tab:datasetmain} cung cấp số lượng hình ảnh, đối tượng của từng lớp của ba bộ dữ liệu.

}


\begin{table}[h]
	\centering
	\begin{threeparttable}
		\caption{Số lượng hình ảnh, đối tượng của tập dữ liệu sử dụng cho nghiên cứu được tổng hợp từ dữ liệu TrashNet, TACO và tự thu thập}
		\begin{tabular}{llrrr}
			\cline{1-5}
			\textbf{Lớp}                           &            & \textbf{TrashNet} & \textbf{TACO} & \textbf{Thu thập} \\ \cline{1-5}
			\multirow{2}{*}{Nhựa - nilon} & Hình ảnh   & 495               & 604           & 392               \\ \cline{2-5}
			                                       & Đối tượng  & 496               & 1252          & 761               \\ \cline{1-5}
			\multirow{2}{*}{Giấy}         & Hình ảnh   & 591               & 349           & 339               \\ \cline{2-5}
			                                       & Đối tượng  & 591               & 488           & 552               \\ \cline{1-5}
			\multirow{2}{*}{Kim loại}     & Hình ảnh   & 409               & 221           & 60                \\ \cline{2-5}
			                                       & Đối tượng  & 410               & 346           & 103               \\ \cline{1-5}
			\multirow{2}{*}{Khác}         & Hình   ảnh & 587               & 365           & 209               \\ \cline{2-5}
			                                       & Đối tượng  & 612               & 547           & 350               \\ \cline{1-5}
		\end{tabular}
		\label{tab:datasetmain}
	\end{threeparttable}
\end{table}


{\fontsize{13}{12} \selectfont

Tập dữ liệu TrashNet được gom các lớp "Cardboard" và "Paper" thành lớp "Giấy", lớp "Glass" và "Trash" thành lớp "Khác" để phù hợp với bốn lớp trong nghiên cứu, sau đó sẽ được gán hộp giới hạn như Hình \ref{fig:dataset_1}.
Tập dữ liệu TACO sử dụng các hộp giới hạn thay cho phân đoạn, tiến hành loại bỏ đi những đối tượng có kích thước quá nhỏ với hình ảnh (dưới 0,05\%), việc này giúp loại bỏ dữ liệu nhiễu và khó phân loại.
Các danh mục của dữ liệu TACO được chuyển đổi sáng các lớp để phù hợp với mô hình được mô tả ở Bảng  \ref{tab:taco_map}.
Trong đó với lớp "Unlabeled litter" sẽ được chú thích nhãn thủ công.
Tập dữ liệu tự thu thập được gán nhãn thủ công dựa vào công cụ trực tuyến Roboflow.
Cuối cùng kết hợp ba tập dữ liệu để hình thành bộ dữ liệu cho nghiên cứu, chia dữ liệu lần lượt cho huấn luyện, xác thực và thử nghiệm là 75\%, 20\% và 5\%, trong đó tập dữ liệu dùng để thử nghiệm gồm hình ảnh rác trong tự nhiên được lấy từ TACO và tự thu thập, thể hiện ở Bảng \ref{tab:datasettest}.

}


\begin{table}[!h]
	\centering
	\begin{threeparttable}
		\caption{Số lượng đối tượng theo từng lớp của tập huấn luyện, xác thực, thử nghiệm dùng trong nghiên cứu}
		\begin{tabular}{lrrr}
			\cline{1-4}
			Lớp          & \textbf{Huấn luyện} & \textbf{Xác thực} & \textbf{Thử nghiệm} \\ \cline{1-4}
			Nhựa - nilon & 1656                & 618               & 235                 \\ \cline{1-4}
			Giấy         & 1179                & 289               & 163                 \\ \cline{1-4}
			Kim loại   & 660                 & 150               & 49                  \\ \cline{1-4}
			Khác         & 999                 & 401               & 109                 \\ \cline{1-4}
		\end{tabular}
	\label{tab:datasettest}
	\end{threeparttable}
\end{table}


\begin{table}[!h]
	\centering
	\begin{threeparttable}
		\caption{Bảng chuyển các lớp của TACO sang lớp của mô hình}
		\begin{tabular}{ll|ll}
			\hline
			\multicolumn{1}{l}{\textbf{TACO}}
			                       & \multicolumn{1}{l}{\textbf{Mô hình}}
			                       & \multicolumn{1}{l}{\textbf{TACO}}
			                       & \multicolumn{1}{l}{\textbf{Mô hình}}                                            \\
			\hline
			Aluminium foil         & Kim loại                             & Magazine paper            & Giấy         \\ \hline
			Battery                & Kim loại                             & Tissues                   & Giấy         \\ \hline
			Aluminium blister pack & Khác                                 & Wrapping paper            & Giấy         \\ \hline
			Carded blister pack    & Khác                                 & Normal paper              & Giấy         \\ \hline
			Other plastic bottle   & Nhựa - nilon                         & Paper bag                 & Giấy         \\ \hline
			Clear plastic bottle   & Nhựa - nilon                         & Plastified paper bag      & Giấy         \\ \hline
			Glass bottle           & Khác                                 & Plastic film              & Nhựa - nilon \\ \hline
			Plastic bottle cap     & Nhựa - nilon                         & Six pack rings            & Khác         \\ \hline
			Metal bottle cap       & Kim loại                             & Garbage bag               & Nhựa - nilon \\ \hline
			Broken glass           & Khác                                 & Other plastic wrapper     & Nhựa - nilon \\ \hline
			Food Can               & Kim loại                             & Single-use carrier bag    & Nhựa - nilon \\ \hline
			Aerosol                & Kim loại                             & Polypropylene bag         & Nhựa - nilon \\ \hline
			Drink can              & Kim loại                             & Crisp packet              & Khác         \\ \hline
			Toilet tube            & Giấy                                 & Spread tub                & Nhựa - nilon \\ \hline
			Other carton           & Giấy                                 & Tupperware                & Nhựa - nilon \\ \hline
			Egg carton             & Giấy                                 & Disposable food container & Nhựa - nilon \\ \hline
			Drink carton           & Giấy                                 & Foam food container       & Giấy         \\ \hline
			Corrugated carton      & Giấy                                 & Other plastic container   & Nhựa - nilon \\ \hline
			Meal carton            & Giấy                                 & Plastic glooves           & Nhựa - nilon \\ \hline
			Pizza box              & Giấy                                 & Plastic utensils          & Nhựa - nilon \\ \hline
			Paper cup              & Giấy                                 & Pop tab                   & Kim loại     \\ \hline
			Disposable plastic cup & Nhựa - nilon                         & Rope \& strings           & Khác         \\ \hline
			Foam cup               & Giấy                                 & Scrap metal               & Kim loại     \\ \hline
			Glass cup              & Khác                                 & Shoe                      & Khác         \\ \hline
			Other plastic cup      & Nhựa - nilon                         & Squeezable tube           & Khác         \\ \hline
			Food waste             & Khác                                 & Plastic straw             & Nhựa - nilon \\ \hline
			Glass jar              & Khác                                 & Paper straw               & Nhựa - nilon \\ \hline
			Plastic lid            & Nhựa - nilon                         & Styrofoam piece           & Khác         \\ \hline
			Metal lid              & Kim loại                             & Unlabeled litter          & Khác         \\ \hline
			Other plastic          & Nhựa - nilon                         & Cigarette                 & Khác         \\ \hline
		\end{tabular}
	\label{tab:taco_map}
	\end{threeparttable}
\end{table}

\bigskip

{\fontsize{13}{12} \selectfont

	Việc gán nhãn cho dữ liệu ảnh hưởng đến hiểu quả mô hình vì rất dễ nhầm lẫn về định nghĩa rác theo thành phần.
	Việc chuyển đổi các danh mục của TACO sang bốn lớp của nghiên cứu gặp phải khó khăn vì có những danh mục có thể nằm ở hai lớp, ví dụ như "Foam food container" có thể là lớp  "Giấy" hoặc "Khác".
	Ngoài ra bộ dữ liệu TACO là do cộng đồng đóng góp, vì vậy việc xuất hiện nhiều lớp "Unlabeled litter" sẽ gây nhiễu cho dữ liệu, cần phải gán nhãn lại.

}

\bigskip

{\fontsize{13}{12} \selectfont
	Ngoài ra việc tăng cường dữ liệu giúp dữ liệu nghiên cứu trở nên đa dạng, tăng tính khái quát của mô hình. Hiện nay các mã nguồn mở hỗ trợ việc tăng cường trong lúc học dữ liệu, vì vậy nghiên cứu sử dụng các tham số như Bảng \ref{tab:thamso}, các tham số đều được cài đặt trong mô hình.

}
\begin{table}[H]
	\centering
	\begin{threeparttable}
		\caption{Các tham số tăng cường dữ liệu cho các mô hình}

		\begin{tabular}{lp{10cm}r}
			\hline
			\textbf{Tên}
			           & \textbf{Mô tả}
			           & \textbf{(\%)}
			\\ \hline
			Lật ngang  & Lật hình ảnh từ trái sang phải, tăng tính đa dạng dữ liệu                                                                                    & 50 \\  \hline
			Lật dọc    & Lật ngược hình ảnh, không ảnh hưởng đến đặc điểm của đối tượng                                                                               & 50 \\  \hline
			Xoay       & Xoay hình ảnh ngẫu nhiên, cải thiện khả năng nhận dạng vật thể ở nhiều hướng                                                                 & 50 \\  \hline
			Xê dịch    & Dịch hình ảnh theo chiều ngang và chiều dọc bằng một phần kích thước hình ảnh, hỗ trợ học cách phát hiện các vật thể nhìn thấy được một phần & 10 \\  \hline
			Phóng đại  & Phóng to, thu nhỏ hình ảnh, mô phỏng hình ảnh với tỉ lệ khác nhau so với hình ảnh chụp                                                       & 50 \\  \hline
			Độ bão hòa & Thay đổi độ bão hòa của hình ảnh một phần, ảnh hưởng đến cường độ màu sắc. Hữu ích cho việc mô phỏng các điều kiện môi trường khác nhau      & 70 \\ \hline
			Độ sáng    & Sửa đổi độ sáng của hình ảnh theo một phần nhỏ, giúp mô hình hoạt động tốt trong nhiều điều kiện ánh sáng khác nhau                          & 40 \\ \hline
		\end{tabular}
	\label{tab:thamso}
	\end{threeparttable}
\end{table}

\section{Các mô hình sử dụng}
\label{sec:model}


{\fontsize{13}{12} \selectfont

	Nghiên cứu sử dụng các mô hình có số lượng tham số nhẹ, tốc độ xử lý nhanh là YOLOv7, YOLOv8, SSD MobileNetv2 thay vì các mô hình họ R-CNN vì các mô hình R-CNN có tham số mô hình lớn, khó khăn trong việc phát triển khi nghiên cứu được cung cấp thêm dữ liệu.
	Ngoài ra nghiên cứu hướng đến việc nhận dạng theo thời gian thực nên sử dụng các mô hình như trên, đáp ứng trên các thiết bị di động với thời gian nhận dạng nhanh.
	Các mô hình họ SSD đạt được sự cân bằng giữa tốc độ và độ chính xác, trong đó mô hình
	SSD MobileNetv2 là mô hình ít tính toán, có kích thước nhỏ và cài đặt dễ dàng với thư viện MMdetection.
	Gần đây những cải tiến của YOLOv8 mang lại như giảm kích thước mạng tích chập từ $6x6$ xuống $3x3$ để trừu tượng hoá hình ảnh hiệu quả, thay thế module C3 bằng module C2f tăng tốc độ xử lí, triển khai dự đoán không cần hộp neo để loại bỏ sự cần thiết của hộp neo,
	tận dụng các kỹ thuật tăng cường dữ liệu khác nhau để cải thiện khả năng tổng quát hóa và giảm tình trạng trang bị quá mức, sử dụng hàm mất mát được sửa đổi dựa trên hàm binary cross-entropy tập trung vào hộp giới hạn và độ tin cậy khi phân lớp.
	Nghiên cứu sử dụng YOLOv7 và YOLOv8 để so sánh độ hiệu quả của hai mô hình mới nhất của họ YOLO, hai mô hình có sự khác biệt về kiến trúc mạng, hàm mất mát và việc sử dụng hay không sử dụng hộp neo.
	Các tham số cài đặt được tối ưu hóa để phù hợp với từng mô hình thể hiện ở Bảng \ref{tab:caidat}.

}


\begin{table}[H]
	\centering
	\begin{threeparttable}
		\caption{Các tham số huấn luyện được cài đặt trong các mô hfnh}

		\begin{tabular}{lrrrr}
			\hline
			\textbf{Mô hình} & \textbf{Batch size} & \textbf{Learning Rate} & \textbf{Tham số} & \textbf{Số hộp neo} \\ \hline
			SSD MobileNetv2    & 16                  & 0,0025                  & 15M              & 6                   \\ \hline
			YOLOv7           & 16                  & 0,01                   & 6M               & 4                   \\ \hline
			YOLOv8           & 16                  & 0,01                   & 3,2M             & 0                   \\ \hline
		\end{tabular}
		\label{tab:caidat}
	\end{threeparttable}
\end{table}

\section{Sơ đồ hệ thống}
\label{sec:sodo}
{\fontsize{13}{12} \selectfont

	Các bước xây dựng hệ thống được chia làm hai giai đoạn được mô tả ở Hình \ref{fig:sodo}:

	\begin{itemize}
		\item Giai đoạn 1: Tiến hành thu thập dữ liệu và kết hợp dữ liệu như đã đề ra ở Mục \ref{sec:dataset}, sau đó dữ liệu được huấn luyện và đánh giá với các mô hình ở Mục \ref{sec:model} để tìm ra mô hình tối ưu cho hệ thống.
		      Qua trình huấn luyện mô hình thực hiện dựa trên các thư viện, dự án như Ultralytics cho YOLOv8, MMdetection cho SSD MobileNetv2 và tài nguyên trên github của chính tác giả YOLOv7.
		\item Giai đoạn 2: Mô hình tối ưu sẽ được sử dụng cho việc xây dựng hệ thống, thông qua mô hình sẽ dự đoán thông tin các loại rác. Từ đó lưu trữ thông tin nhãn và tọa độ giúp bản đồ hiển thị thông tin cần thiết.
		      Dữ liệu sẽ được lưu trữ ở MongoDB.
	\end{itemize}

}

\begin{figure}[H]
	\centering
	\includegraphics[width=1\textwidth]{sodo.png}
	\caption{Sơ đồ hệ thống}
	\label{fig:sodo}
\end{figure}

\end{document}



\documentclass[../the.tex]{subfiles}


\begin{document}

\section{Kết luận}

{\fontsize{13}{12} \selectfont
Nghiên cứu của chúng tôi cho thấy rằng các mô hình dựa trên YOLOv4 có thể hoạt động tốt hơn mô hình U-Net trên một tập dữ liệu phức tạp về hình ảnh X-quang cho mục đích phát hiện gãy xương cổ tay. Phân tích về kết quả đầu ra của các phương pháp đã chỉ ra rằng bản đồ nhiệt có thể bị mờ trong khi các hộp giới hạn chính xác hơn trong việc cung cấp vùng đứt gãy chính xác. Do đó, có thể nếu hợp nhất hai mô hình để tạo ra hộp giới hạn xung quanh vết nứt cũng như bản đồ nhiệt của vết nứt. Bằng cách đó, chúng tôi sẽ có thể thu được những điều tốt nhất của cả hai mô hình: số lượng vết gãy chính xác từ các hộp giới hạn và phát hiện vết gãy thông minh theo pixel bên trong hộp giới hạn. Một cách tiếp cận khác là phát triển một mô hình duy nhất sẽ thực hiện đồng thời cả hai. Hơn nữa, sự kết hợp giữa U-Net và YOLOv4, cùng với việc xử lý ngôn ngữ tự nhiên sẽ giúp chúng ta xây dựng một hệ thống CAD để chẩn đoán gãy xương cổ tay có thể tạo ra các báo cáo chi tiết, và có thể được sử dụng trong thực hành lâm sàng.
}
\bigskip

\section{Hạn chế và đề xuất}

{\fontsize{13}{12} \selectfont
Hiện tại, mô hình của chúng tôi chỉ hoạt động được trên ảnh X-Quang, chưa thể hoạt động trên ảnh CT. Chúng thôi sẽ cải thiện vấn đề này bằng cách xây dựng một thuật toán có thể chuyển ảnh CT sang ảnh X-Quang hoặc thu thập thêm ảnh CT để huấn luyện lại mô hình. 

Chúng tôi sẽ tiến hành thử nghiệm các mô hình phát hiện đối tượng khác để so sánh hiệu năng. Ngoài ra, thu thập thêm các hình ảnh X-Quang về gãy xương ở những vị trí khác (ngoài cổ tay) để mở rộng tập dữ liệu và huấn luyện lại các mô hình trước đó để đạt được sự đa dạng hóa trong vấn đề phát hiện gãy xương.
}
\end{document}
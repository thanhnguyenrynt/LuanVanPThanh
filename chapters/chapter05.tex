\documentclass[../the.tex]{subfiles}


\begin{document}

\section{Kết luận}

% {\fontsize{13}{12} \selectfont
% Nghiên cứu của chúng tôi cho thấy rằng các mô hình dựa trên YOLOv4 có thể hoạt động tốt hơn mô hình U-Net trên một tập dữ liệu phức tạp về hình ảnh X-quang cho mục đích phát hiện gãy xương cổ tay. Phân tích về kết quả đầu ra của các phương pháp đã chỉ ra rằng bản đồ nhiệt có thể bị mờ trong khi các hộp giới hạn chính xác hơn trong việc cung cấp vùng đứt gãy chính xác. Do đó, có thể nếu hợp nhất hai mô hình để tạo ra hộp giới hạn xung quanh vết nứt cũng như bản đồ nhiệt của vết nứt. Bằng cách đó, chúng tôi sẽ có thể thu được những điều tốt nhất của cả hai mô hình: số lượng vết gãy chính xác từ các hộp giới hạn và phát hiện vết gãy thông minh theo pixel bên trong hộp giới hạn. Một cách tiếp cận khác là phát triển một mô hình duy nhất sẽ thực hiện đồng thời cả hai. Hơn nữa, sự kết hợp giữa U-Net và YOLOv4, cùng với việc xử lý ngôn ngữ tự nhiên sẽ giúp chúng ta xây dựng một hệ thống CAD để chẩn đoán gãy xương cổ tay có thể tạo ra các báo cáo chi tiết, và có thể được sử dụng trong thực hành lâm sàng.
% }
% \bigskip
{\fontsize{13}{12} \selectfont

Nghiên cứu cho thấy mô hình họ YOLO hoạt động tốt hơn so với SSD, tốt nhất với mô hình YOLOv8 trên tập dữ liệu phức tạp về hình ảnh rác ở tự nhiên, đặc biệt là rác trên đường phố.
YOLOv8 không những cho thấy khả năng vượt trội trong việc phát hiện đối tượng rác mà còn hiệu quả trong nhiệm vụ phân loại rác thành phần.
Mô hình nhận dạng tốt các đối tượng là kim loại và có kết quả không khả quan với lớp rác khác do ảnh hưởng bởi nền xung quanh đối tượng và sự đa dạng trong lớp.
So với các nghiên cứu trước đây chủ yếu thực hiện việc phân lớp có đặc tính dữ liệu được chụp ở môi trường có ánh sáng, nền lí tưởng nên không thể áp dụng với các hình ảnh được chụp trực tiếp ngoài tự nhiên, thì nghiên cứu này đạt được độ chính xác tương đối trên tập dữ liệu thử nghiệm.
Kết quả trên của mô hình YOLOv8 được áp dụng để xây dụng lên hệ thống phát hiện và phân loại rác thải trên đường phố, lập ra bản đồ phân bố bốn loại rác thải.

}

\bigskip

{\fontsize{13}{12} \selectfont 
Việc kết hợp các tập dữ liệu TACO, TrashNet, tự thu thập giúp hạn chế việc mất cân bằng dữ liệu, tăng tính đa dạng cho việc huấn luyện.
Sau quá trình tìm hiểu và thực nghiệm, nghiên cứu cho thấy việc áp dụng các ảnh nền không chứa nhãn giúp cải thiện mô hình. Ngoài ra việc phân chia thành các lớp có thể ảnh hưởng đến kết quả của việc phân loại.

}
\section{Hạn chế và đề xuất}
\subsection{Hạn chế}
{\fontsize{13}{12} \selectfont
Hiện tại mô hình có cải thiện so với các mô hình trước đây về phát hiện rác trong tự nhiên, tuy nhiên vẫn còn nhiều hạn chế như độ chính xác của mô hình chưa cao, mô hình còn nhận dạng nhằm các đối tượng không phải là rác, các mô hình khó nhận dạng các đối tượng nhỏ. Mô hình chưa đủ dữ liệu để phân loại, cần bổ sung thêm dữ liệu. Hệ thống chưa lấy được tọa độ của video nên chỉ đang sử dụng nhận dạng với hình ảnh. Việc gán nhãn thủ công và chuyển đổi nhãn của tập dữ liệu TACO có thể phát sinh nhiều sai sót ảnh hưởng quá trình học của mô hình.	

}
\subsection{Đề xuất}
{\fontsize{13}{12} \selectfont 
Với những hạn chế đã đề cập, chúng tôi đề xuất việc tăng cường thu thập dữ liệu cho mô hình, dữ liệu cần thu thập đa dạng hơn để cải thiện khả năng nhận dạng các vật nhỏ, đa dạng hơn về các môi trường khác như trên mặt nước. Ý tưởng 
thu thập bằng việc gắn camera lúc di chuyển hoặc hệ thống CCTV được đặt trên đường phố hay các con sông.
Các hình ảnh của người dùng khi sử dụng hệ thống cũng được bổ sung vào tập dữ liệu để huấn luyện. 
Nghiên cứu nên phát triển chức năng xác định vị trí khi quay video để sử dụng cho việc phát hiện theo thời gian thực, giúp cải thiện dữ liệu cho bản đồ phân bố.
Ngoài ra cần phát triển thêm ứng dụng di động, mở rộng hệ thống sử dụng cho cộng đồng, phát triển chức năng ghi nhận phản hồi từ người dùng như việc đánh giá rác đơn lẻ hay bãi rác, mức độ quan trọng của hình ảnh và chỉnh sửa, gán nhãn thêm cho các rác mà mô hình chưa thể phát hiện được.
}
\end{document}
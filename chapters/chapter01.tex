\documentclass[../the.tex]{subfiles}


\begin{document}


\section{Đặt vấn đề}
\label{tong_quan}

{\fontsize{13}{12} \selectfont

Ô nhiễm rác là tình trạng lượng rác thải sinh hoạt, công nghiệp, và các loại chất thải khác được thải ra môi trường vượt quá khả năng tự phân hủy và xử lý của tự nhiên. Từ đó, chúng gây ảnh hưởng và làm ô nhiễm không khí ở các khu vực xung quanh. Ngoài ra tình trạng này cũng thể hiện khi rác thải được đốt cháy khiến các khí thải độc hại, khói bay ra gây ô nhiễm không khí và ảnh hưởng xấu tới sức khỏe con người, hệ động, thực vật.
Tình trạng này ngày càng trở nên nghiêm trọng do sự gia tăng dân số, tốc độ đô thị hóa, và thói quen tiêu dùng không bền vững. Việc giải quyết ô nhiễm rác đòi hỏi sự chung tay từ các cấp chính quyền, doanh nghiệp và cộng đồng, cũng như việc áp dụng các công nghệ hiện đại và chính sách quản lý hiệu quả. 

}

\bigskip

{\fontsize{13}{12} \selectfont

Ô nhiễm môi trường bằng rác thải vẫn đang là vấn đề gây nhức nhối ở toàn cầu.
Phần lớn rác thải ở nhiều quốc gia và vùng lãnh thổ được chôn lấp hoặc đốt, dẫn đến ô nhiễm đất và không khí, nhiều bãi chôn lấp không đáp ứng các tiêu chuẩn kỹ thuật, gây ra nguy cơ ô nhiễm nghiêm trọng.
Năm 2023, thế giới sản xuất khoảng 2,1 tỷ tấn chất thải rắn đô thị, dự kiến đến năm 2050 sản lượng chất thải tăng lên 3,8 tỷ tấn\footnote[1]{https://www.unep.org/resources/global-waste-management-outlook-2024} (xem Hình \ref{fig:thongke_rac}).
Lượng rác thải trung bình mỗi người một ngày là 0,74 kg, dao động từ 0,11 đến 4,54 kg. Mặc dù chỉ chiếm 16\% dân số thế giới nhưng các quốc gia có thu nhập cao tạo ra khoảng 34\% tương đương 683 triệu tấn chất thải trên toàn thế giới.
Thành phần chất thải cũng có sự khác nhau giữa các quốc gia, ở các nước thu nhập cao tạo ra ít chất thải hữu cơ (32\%) nhưng nhiều chất thải khác như nhựa, giấy, bìa cứng, kim loại và thủy tinh (51\%). Trong khi đó ở các nước thu nhập thấp thì các chất thải có thể tái chế chỉ chiếm 20\%.
Việc quản lý và thu gom chất thải ở các nước thu nhập thấp chỉ khoảng 48\% ở thành thị và 20\% ở nông thôn dẫn đến tình trạng tồn động rác và hình thành nên các bãi rác bất hợp pháp, gây ô nhiễm môi trường ở khu vực\footnote[2]{https://datatopics.worldbank.org/what-a-waste}. 
Theo thống kê ở những nước có tốc độ đô thị hóa nhanh và dân số đông thì lượng rác thải sản sinh ra ngày càng nhiều. Ngày nay, Trung Quốc và Mỹ là những nước đứng đầu về việc ô nhiễm môi trường từ rác thải.
Với khoảng 18\% dân số thế giới, Trung Quốc là nơi tạo ra số lượng rác lớn nhất toàn cầu, chiếm hơn 15\% tổng số. Tuy nhiên với lượng rác trên mỗi người thì Mỹ mới là quốc gia đứng đầu, với 2,58 kg. Tiếp theo là Canada (2,33 kg/người) và Úc (2,23 kg/người).
Là một trong những khu vực bùng nổ về tốc độ đô thị hóa, Đông Nam Á cũng đồng thời ghi nhận lượng rác thải tăng lên đáng kể.
Trong đó, điển hình là Indonesia khi quốc gia này đã và đang đối mặt với các vấn đề nghiêm trọng về rác trong suốt nhiều năm, trung bình Indonesia thải ra hàng triệu tấn rác mỗi ngày\footnote[3]{https://worldpopulationreview.com/country-rankings/plastic-pollution-by-country}.

}

\begin{figure}[H]
	\centering
	\includegraphics[width=1\textwidth]{thongke_rac.png}
	\caption{Dự báo lượng chất thải đô thị toàn cầu phát sinh vào các năm 2030, 2040 và 2050}
	\label{fig:thongke_rac}
\end{figure}

{\fontsize{13}{12} \selectfont

Ở Việt Nam, hằng năm thải ra hơn 28 triệu tấn rác thải, tương đương 0,3 kg bình quân mỗi người một ngày, với 76\% trong đó được ném ra các bãi rác tập trung. Sự thiếu phân loại từng loại rác, cùng với vấn đề của chất hữu cơ và độ ẩm cao khiến cho việc tái chế chất thải hỗn hợp thành nguyên liệu thô hoặc thành năng lượng trở nên khó khăn, điều này phần nào giải thích được tại sao các bãi rác thải tập trung có lượng rác chưa được xử lí với tỉ lệ cao\footnote[1]{https://woimacorporation.com/ngap-trong-rac-thai-van-de-dang-dien-ra-o-viet-nam}. 
Ở đô thị, mọi người dễ dàng bắt gặp được tình trạng các túi rác, ni lông, chai nhựa nằm rải rác trên các vỉa hè hay thậm chí dưới lòng đường.
Các rác này nếu không được xử lý sớm lâu ngày sẽ hình làm thành các bãi rác hoặc tích tụ trong hệ thống thoát nước làm ngập đường phố.
Điều đó vừa gây mất cảnh quan đô thị vừa ảnh hưởng xấu đến chất lượng đời sống. 
Việt Nam đã phát triển một kế hoạch tập trung vào vấn đề tái chế và xử lí chất thải vào năm 2025 với hi vọng sẽ thu gom và xử lí tới 90\% chất thải rắn tại các thành phố, cùng với đó là tái chế hoặc tái sử dụng 85\% của chất thải rắn đế sản xuất năng lượng hoặc phân bón hữu cơ.
Các thành phố lớn của Việt Nam đều có hệ thống thu gom rác thải nhưng ở khu vực nông thôn phần lớn rác thải bị đổ xuống sông, suối hoặc các bãi rác tự phát mà không được xử lí đúng cách\footnote[2]{https://woimacorporation.com/ngap-trong-rac-thai-van-de-dang-dien-ra-o-viet-nam}. 

}

\bigskip

{\fontsize{13}{12} \selectfont

Ở ĐBSCL ước tính có đến 30\% lượng rác thải không được quản lý hoặc được người dân và các cơ sở sản xuất tự xử lý bằng cách xả thải trực tiếp xuống các ao sông, kênh rạch.
Tổng lượng rác toàn vùng Đồng bằng sông Cửu Long khoảng 4.200 tấn/ngày. Trong đó, lượng thu gom từ các thành phố thị trấn khoảng 3.200 tấn/ngày, đạt tỷ lệ 70-80\%. Số còn lại không được quản lý, được người dân, các cơ sở không tuân thủ theo các quy định của các đô thị, tự xả xuống các sông, kênh rạch gây ô nhiễm các sông rạch, ảnh hưởng đến môi trường chung trong khu vực.
Trong đó đáng báo động nhất là Thành phố Cần Thơ, mỗi ngày người dân ở đây thải ra 650 tấn chất thải rắn sinh hoạt nhưng tỷ lệ thu gom chưa tới 70\%, hơn 30\% còn lại được xử lý không đúng cách.
Hầu hết các địa phương ở ĐBSCL đều có bãi rác quy mô nhỏ, tuy nhiên đều là các bãi chôn lấp không hợp vệ sinh, chỉ chôn lấp tự nhiên chưa đúng quy trình dẫn đến tình trạng ô nhiễm không khí trong khu vực\footnote[1]{https://congnghiepmoitruong.vn/thuc-trang-rac-thai-dang-bao-dong-o-dong-bang-song-cuu-long-12023.html}.
Việc chính quyền địa phương nắm được tình trạng phân loại, xả rác, các bãi rác bất hợp pháp hỗ trợ cho việc xử lý rác thải kịp thời, giảm số lượng rác đi vào bãi rác. 

}

\bigskip

{\fontsize{13}{12} \selectfont
Với sự phát triển mô hình đô thị thông minh, hệ thống nhận dạng và phân loại rác thải là một bước tiến quan trọng trong việc quản lý để xây dựng một thành phố xanh, sạch, đẹp.
Việc áp dụng thị giác máy tính vào phân loại rác thải dựa trên những hình ảnh trên đường phố sẽ giúp rút ngắn thời gian phát hiện và phân loại rác, từ đó lập ra bản đồ số giúp thống kê tình trạng phân bố của từng loại rác thải ở từng địa phương. Nghiên cứu và triển khai hệ thống nhận dạng và phân loại rác thải là cơ hội để thúc đẩy sự sáng tạo công nghệ trong lĩnh vực môi trường và quản lý đô thị, mang lại nhiều lợi ích thực tiễn cho cộng đồng và doanh nghiệp.

}

\bigskip

{\fontsize{13}{12} \selectfont 

Hiện nay, với sự phát triển của học sâu và ngân hàng các bộ dữ liệu về rác cũng rất phổ biến, có 

nhiều nghiên cứu cho thấy hiệu quả trong vấn đề phân lớp \cite{yang2016classification} \cite{shah2022method} \cite{ahmad2020intelligent}. 
Tuy nhiên vấn đề đặt ra là việc rác thải ở tự nhiên có thể bị biến dạng, phân mảnh hoặc bị chồng chéo lên nhau.
Các kết cấu tự nhiên như cát, đá, cỏ cây, vật thể trộn lẫn với rác làm cho các nghiên cứu về vấn đề này đạt kết quả ở mức tạm chấp nhận \cite{Majchrowska_2022} \cite{9122693} \cite{8793975} \cite{proença2020taco}.
Việc phát hiện rác thải ngoài tự nhiên là một thách thức so với những rác thải ở băng chuyền với điều kiện nhận dạng lý tưởng.

}


\section{Mục tiêu của đề tài}
\label{muc_tieu}


{\fontsize{13}{12} \selectfont

Mục đích đề tài nhắm đến xây dựng hệ thống hỗ trợ phát hiện và phân loại rác với đầu vào là hình ảnh trên đường phố. Kết quả được đạt được là hình ảnh đã được xác định vùng là vật thể rác và tên loại rác, từ đó xây dựng lên bản đồ phân bố. Cụ thể nghiên cứu thực hiện nhưng nội dung sau:

\begin{itemize}
  \item Tiến hành thu thập, gán nhãn tập dữ liệu cá nhân. Tìm hiểu tập dữ liệu TrashNet \cite{yang2016classification} và TACO \cite{proença2020taco}. Tiền xử lí, kết hợp các tập dữ liệu để tạo ra bộ dữ liệu cho việc phát hiện đối tượng.
  Đánh giá các mô hình như YOLO, SSD trên bộ dữ liệu ngoài tự nhiên trong nhiệm vụ phát hiện và phân loại rác. Sau đó so sánh độ hiệu quả của mô hình và tìm ra mô hình tối ưu để xây dựng hệ thống.
  
  \item Xây dựng ứng dụng nhận đầu vào là hình ảnh rác ở đường phố, hình ảnh được mô hình nhận dạng, phân loại rác, sau đó ghi nhận vào cơ sở dữ liệu các thông tin vị trí, vùng phát hiện rác, loại rác để hiển thị lên bản đồ phân bố.
 
\end{itemize}

}

{\fontsize{13}{12} \selectfont

Qua các thực nghiệm, phát hiện đối tượng là một cách tiếp cận mới phù hợp để phân loại rác thải trong môi trường tự nhiên, việc bổ sung các hình ảnh không có rác giúp mô hình cải thiện hiệu suất. 
Nghiên cứu dựa vào tình hình dữ liệu ở ĐBSCL và các quy định của Luật Bảo vệ môi trường 2020 và ý tưởng phân loại rác theo thành phần của dữ liệu TrashNet để phân loại rác thành bốn lớp: kim loại, nhựa - nilon, giấy, rác khác. Từ đó có thể áp dụng thực tiễn để phát triển bản đồ phân bố rác ở địa phương. 

}


\section{Phạm vi của luận văn}
\label{pham_vi}

{\fontsize{13}{12} \selectfont

Từ mục tiêu của đề tài, luận văn tập trung nghiên cứu các nhiệm vụ sau:

\begin{itemize}
  \item Tìm hiểu các mô hình.
  \item Cài đặt các thư viện cần thiết.
  \item Huấn luyện, xây dựng mô hình phát hiện và phân loại rác.
  \item Đánh giá các mô hình và tìm ra mô hình tối ưu nhất.
  \item Xây dựng ứng dụng để nhận dạng rác và hiển thị bản đồ.
\end{itemize}

}

\section{Bố cục luận văn}
\label{bo_cuc}

{\fontsize{13}{12} \selectfont

Các chương tiếp theo của luận văn được tổ chức như sau:

\begin{itemize}
  \item \textbf{Chương 2. Cơ sở lí thuyết} trình bày về các nghiên cứu liên quan. Kết hợp trình bày lý thuyết học sâu, các thành phần liên quan, tổng quan về các mô hình nhận dạng đối tượng.
  \item \textbf{Chương 3. Phương pháp nghiên cứu} mô tả chi tiết về phương pháp nghiên cứu với mô hình YOLOv7, YOLOv8, SSD MobileNetv2.
  \item \textbf{Chương 4. Kết quả thực nghiệm và thảo luận} trình bày các kết quả đạt được dựa vào các phương pháp đánh giá.
  \item \textbf{Chương 5. Kết luận và đề xuất} tổng kết lại các kết quả đạt được trong quá trình thực hiện, đưa ra các hạn chế và hướng phát triển.
\end{itemize}

}

 % {\fontsize{13}{12} \selectfont
% \begin{figure}[H]
% \centering
% \tikzset{every picture/.style={line width=0.75pt}} %set default line width to 0.75pt        

% \begin{tikzpicture}[x=0.75pt,y=0.75pt,yscale=-1,xscale=1]
% %uncomment if require: \path (0,300); %set diagram left start at 0, and has height of 300

% %Image [id:dp5541696886497023] 
% \draw (155.25,139.5) node  {\includegraphics[width=130pt,height=110pt]{face.jpg}};
% %Image [id:dp6999670900516217] 
% \draw (504.25,139.5) node  {\includegraphics[width=130pt,height=110pt]{cnn.jpg}};
% %Right Arrow [id:dp363270476880974] 
% \draw  [fill={rgb, 255:red, 0; green, 255; blue, 73 }  ,fill opacity=1 ] (299,138) -- (341,138) -- (341,158) -- (369,148) -- (341,168) -- (341,158) -- (299,158) -- cycle ;

% \end{tikzpicture}
% 	\caption{Phát hiện đối tượng thông thường (Nguồn: www.cgtrader.com)}
% 	\label{fig:detect}
% \end{figure}}

% \bigskip




\end{document}
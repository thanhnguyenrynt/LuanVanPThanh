\documentclass[../the.tex]{subfiles}
\begin{document}

\section*{1. Nhận xét kết quả đạt được:}
\label{comment}
Sau thời gian nghiên cứu, tìm hiểu về đề tài dưới sự hỗ trợ của thầy cô bạn bè và đặc biệt là giáo viên hướng dẫn thầy Thái Minh Tuấn, báo cáo cơ bản đã được hoàn thành và đạt kết quả như sau:
\bigskip

-	Phần Lý thuyết:

	+ Hiểu được các thành phần của một mạng tích chập đầy đủ.
	
	+ Những ưu điểm và nhược điểm của mạng tích chập đầy đủ.
	
	+ Những kỹ thuật tính toán cao cấp như tích chập, lan truyền ngược, trường điều kiện ngẫu nhiên.
	
	+ Biết cách áp dụng máy học vào thực tiễn.
	
	+ Sử dụng Python và thư viện làm việc chuyên dụng để xây dựng một ứng dụng có hơi hướng học sâu nói riêng và trí tuệ nhân tạo nói chung.
	
\bigskip

-	Phần demo:.

	+ Phần giao diện trình bày gọn gàng với giao diện thân thiện.
	
	+ Ứng dụng cho phép người dùng sửa đổi một số thông số quan trọng của trường điều kiện ngẫu nhiên.
	
	+ Hình nền sử dụng thay thế sau khi xóa phông thay đổi dễ dàng, cho phép chụp hình và cho phép chọn hình ảnh từ máy tính để thực hiện tách phông
	
\bigskip



\section*{2. Hạn chế:}
\label{limit}
Mặc dù với sự cố gắng hết sức nhưng vẫn không thể tránh khỏi những sai sót như sau:

-	Lý thuyết:

	+ Sử dụng giải thuật chưa được tối ưu.
	
	+ Chưa trình bày sự so sánh giữa các giải thuật phân vùng ngữ nghĩa như SegNet, Unet, DeepLab,...
	
	+ Thiếu sót trong việc trình bày nội dung dùng để tối ưu bằng \gls{hmm}, \gls{memm},... so với \gls{crfs}.
\bigskip
	

-	Phần demo: 

	+ Giao diện chưa có sự chăm chút tỉ mỉ.
	
	+ Ứng dụng chỉ chạy được cả \gls{fcn} và \gls{crfs} cho từng tấm ảnh, không thể thực hiện cho video hay thời gian thực do cấu hình phần cứng.
	
	+ Các tính năng chưa được tiện ích và tối ưu.
	
	+	Tập dữ liệu chưa được phong phú vì chỉ phân vùng được con người.

\bigskip

\section*{3. Hướng phát triển:}
\label{develop}
Hướng phát triển của bài báo cáo gồm:

	-	Sử dụng các giải thuật cao cấp hơn đã nêu và thay đổi cách lập trình bằng tensorflow để tối ưu quá trình tính toán.
	
	-	Tích hợp với thiết bị di động để tạo nên những ứng dụng tiện ích thực sự.
	
	-	Thay đổi tập dữ liệu phổ biến cả về không gian và ánh sáng đồng thời kết hợp với các loại ảnh nền độc đáo khác nhau.
\bigskip
	


\end{document}